\documentclass[10pt]{article}
\usepackage[utf8]{inputenc}
\usepackage{times} 
\usepackage[a4paper,margin=1.35in]{geometry}
\usepackage{graphicx}
\usepackage{amsmath}
\usepackage{booktabs}
\usepackage{multirow}
\usepackage{url}
\usepackage{hyperref}
\usepackage{listings}
\usepackage{color}

\definecolor{codegreen}{rgb}{0,0.6,0}
\definecolor{codegray}{rgb}{0.5,0.5,0.5}
\definecolor{codepurple}{rgb}{0.58,0,0.82}
\definecolor{backcolour}{rgb}{0.95,0.95,0.92}

\lstdefinestyle{mystyle}{
    backgroundcolor=\color{backcolour},   
    commentstyle=\color{codegreen},
    keywordstyle=\color{codepurple},
    numberstyle=\tiny\color{codegray},
    stringstyle=\color{codegreen},
    basicstyle=\ttfamily\footnotesize,
    breakatwhitespace=false,         
    breaklines=true,                 
    captionpos=b,                    
    keepspaces=true,                 
    numbers=left,                    
    numbersep=5pt,                  
    showspaces=false,                
    showstringspaces=false,
    showtabs=false,                  
    tabsize=2
}

\lstset{style=mystyle}

\hypersetup{
    colorlinks=true,
    linkcolor=blue,
    filecolor=magenta,      
    urlcolor=cyan,
    pdftitle={Requirements for Configuration Performance Learning for X264},
    pdfpagemode=FullScreen,
}

\singlespacing 
\title{Requirements for Configuration Performance Learning for X264}
\author{Configuration Performance Learning Research}
\date{\today}

\begin{document}

\maketitle

\section{Software Requirements}

This project requires the following software components to run the machine learning models for predicting x264 configuration performance:

\subsection{Python Environment}
The implementation requires Python 3.6 or higher. We recommend using a virtual environment to manage dependencies.

\begin{lstlisting}[language=bash]
# Create and activate a virtual environment
python -m venv x264_env
source x264_env/bin/activate  # On Windows: x264_env\Scripts\activate
\end{lstlisting}

\subsection{Required Python Packages}
The following Python packages are required to run the code. Install them using pip:

\begin{lstlisting}[language=bash]
pip install numpy pandas scikit-learn lightgbm
\end{lstlisting}

\begin{table}[h]
\centering
\caption{Required Python Package Versions}
\begin{tabular}{lll}
\toprule
\textbf{Package} & \textbf{Min Version} & \textbf{Purpose} \\
\midrule
numpy & 1.19.0 & Numerical operations and array handling \\
pandas & 1.1.0 & Dataset loading and preprocessing \\
scikit-learn & 0.24.0 & Linear Regression model and metrics \\
lightgbm & 3.2.0 & Gradient boosting implementation \\
\bottomrule
\end{tabular}
\end{table}

\subsection{Hardware Requirements}
The experiments were conducted on a system with the following specifications:

\begin{itemize}
    \item CPU: Intel Core i7 or equivalent (4+ cores recommended)
    \item RAM: Minimum 8GB (16GB recommended for large datasets)
    \item Disk Space: At least 1GB free space for datasets and result logs
\end{itemize}

While the models can run on less powerful hardware, training time may increase significantly.

\section{Directory Structure}
The codebase expects a specific directory structure to function properly:

\begin{lstlisting}
config-performance-learning-x264/
├── data_loader.py
├── lightGBM.py
├── lr.py
├── datasets/
│   └── x264/
│       ├── blue_sky_1080p25.csv
│       ├── Johnny_1280x720_60.csv
│       └── ... (other video datasets)
└── log/
    └── YYYYMMDD/
        ├── lightgbm/
        └── linear-regression/
\end{lstlisting}

The `datasets` directory contains subdirectories for each system (x264 in this case), with CSV files for each video. The `log` directory stores results organized by date and model type.

\section{Dataset Format}
Each dataset CSV file should have the following format:

\begin{itemize}
    \item Rows represent different configurations of x264
    \item Columns represent configuration parameters
    \item The last column contains the performance metric (runtime in seconds)
    \item No header row (or will be automatically skipped if present)
\end{itemize}

The dataset used in this study is based on x264 version baee400 and includes 3,113 unique configurations across 25 parameters.

\section{Environment Variables}
No specific environment variables are required for this project.

\section{Compilation}
As this is a Python project, no compilation is needed. The scripts can be run directly using the Python interpreter.

\end{document}
